\documentclass{article}
\usepackage[utf8]{inputenc}
\usepackage[document]{ragged2e}

\usepackage{amsmath}
\newcommand*{\Comb}[2]{{}^{#1}C_{#2}}%
{\sfCS6845: MTTOC \hfill Name: MANOJ KALASKAR}
\vspace{3mm}\\ 
\noindent 
{\sf Scribe\#1 \hfill Roll No: CS13M027}
\vspace{3mm}

\hrule
\vspace{7mm}

\begin{document}

\section{Hypergraph colouring}
\par

Hypergraph is a graph where an edge can be connected to any number of vertices(and not just two). We define hypergraph formally as 
     $H(V,E)$ where V is a set of vertices and E is set of edges where $e\in E$ $\subseteq P(V)\setminus \phi$ where $P(V)$ is a power set of $V$. \\
     
\textbf {Definations : }

\begin{itemize}
  \item A hypergraph is said to be $n$-$uniform$ if every edge contains exactly $n$ vertices.
  \item A hypergraph colouring is said to be valid  iff there is no  monochromatic edge with cardinality atleast 2. A hypergraph is $k$-$colourable$ if there exist colouring, using upto $k$ colours.   
  \end{itemize}
\\
\textbf{Goal: }
Given $n$, we want to compute the smallest number of hyper edges
for which there exits a $n$-$uniform$ hypergraph which is NOT $2$-$colourable$(means which is monochromatic),we label it $m(n)$.
\\
\\

\textbf{Lower Bound on m(n): }

$Claim$: All $n$-$uniform$ hypergraphs H with$|E(H)|$$<$ $2^{n-1}$ are 2-$colourable$. Hence $m(n) \geq 2^{n-1}$.

$Proof :$  To show that 2-$colouring$ is possible, it is enough to show that 2-$colouring$ exits.

Let the $n$-$regular$ hypergraph with $|E|<2^{n-1}$.\\

\begin{centering}
\vspace{5mm}
Pr(graph is 2-colourable) $>$ 0 \\
\end{centering}

which is equivalent of

\begin{centering}
Pr(graph is monochromatic) $<$ 1
\end{centering}
\vspace{5mm}
\begin{centering}

Pr( H is monochromatic) = Pr(each edge is monochromatic )\\
\vspace{7mm}
 \hspace{40mm} \leq \sum\limits_{e\in E(H)} Pr($e is monochromatic$)
 \vspace{7mm}
 \hspace{10mm} $=$ \sum\limits_{e\in E(H)} \frac{2}{2^{n}}
  \newpage
 \hspace{35mm} $=$  $\frac{|E(H)|}{2^{n-1}}$ $<$ 1 (as $|E(H)| < 2^{n-1}.)$\\
\end{centering}
\vspace{7mm}
 So, this results proves that there exits 2-$colouring$.
 \par
 \newline
 \vspace{10mm} \hspace{5mm}
 \textbf{Upper Bound on m(n):}
 \par
 We have to find the maximum m(n) for which the given graph is monochromatic.
 \par
Here instead of fixing the hypergraph(like in the previous claim), we will fix the colouring. We show that randomly constructed hypergraph with $|V|$ vertices has the required upper bound and then find the optimal bound.
\\
Let us pick vertices independently and uniformly at random. We form $m$ hyperedges of size $n$. We have  $ \Comb{|V|}{n}$ possible edges.

We want to show
Pr_E [ $\exists$ a colouring such that no hyperedge in $E$ is monochomratic] $<$ 1\\


Now lets analyze for fixed colouring.\\

\begin{centering}

$Pr(m)$=$Pr$[A hyperedge picked is monochromatic] \\
=$\frac{\Comb{|V_R|}{n}+\Comb{|V_B|}{n}}{\Comb{|V|}{n}}$
$\geq$ $\frac{\Comb{\frac{|V|}{2}}{n}} {\Comb{|V|}{n}}$
\end{centering}
\\
Pr[hyperedge non-monochromatic] $\leq 1-P(m)$
\\
Pr[No hyperedge is monochromatic]$\leq (1-P(m))^{m}$
\\
Hence,
\\ Pr[ $\exists$ colouring such that no hyperedge in $H$ is monochromatic]\\ $\leq$ No. of Colourings * Pr[No hyperedge in $H$ is monochromatic]  ..(Using Union Bound)
\\= 2^{|V|}(1-P(m))^{m}
\\
We want: 
\begin{centering}
\\ 2^{|V|}(1-P(m))^{m} < 1\\
$\implies$


\\
\vspace{3mm}
(1-$P(m)$)^{m} < \frac {1}{2^{|V|}}
\\
\vspace{3mm}

Using (1-$P(m)$)\leq $e^{-P(m)}$,\\
\vspace{3mm}

$e^{-mP(m)})$ $<$ $\frac{1}{2^{|V|}}$\\
\vspace{3mm}

2^{|V|} < $e^{mP(m)}$
\vspace{3mm}

\\ Taking $\ln$ both sides,
\vspace{3mm}

\\ $|V|\ln2 < mP(m)$
\vspace{3mm}


\\ $m > \frac{|V|\ln 2}{P(m)}$
\\and \hspace{2mm} 
\vspace{3mm}

$ m \leq  \frac{|V|\ln 2}{P(m)}$ * $\Comb{|V|}{n}$
\vspace{3mm}

$ m \leq  \frac{|V|\ln 2}{\Comb{\frac{|V|}{2}}{n}}$ * $\Comb{|V|}{n}$

\end{centering}
\newline
By choosing $|V|$ that minimizes $m$, we get:
\\ $m\leq  n^{2}*2^{n}$  giving us an upper bound on $m(n)$.

\end{document}
